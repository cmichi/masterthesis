\documentclass[tikz]{standalone}

\usepackage{fontspec}
\usepackage{xunicode}
\usepackage{xltxtra}

\usepackage{polyglossia}
\setdefaultlanguage{english}   

\usepackage[misc]{ifsym}
\usepackage{amsmath} 

\defaultfontfeatures{
        Scale=MatchLowercase, % match the lowercase 'x' in different fonts
        Mapping=tex-text,   
        %Ligatures={TeX, Common, Rare},
        Numbers={OldStyle, Proportional}
}
\setmainfont{Linux Libertine}

% tikz defintions
\usepackage{tikz}
\usetikzlibrary{fit}
\usetikzlibrary{arrows.meta}
\usetikzlibrary{positioning}
\usetikzlibrary{shapes.geometric}
\usetikzlibrary{decorations.pathreplacing}
\usetikzlibrary{calc} 
\usetikzlibrary{patterns}

\tikzstyle{roundBox} = [text width=6em, fill=white, minimum height=5em, rounded corners, text centered, draw=black]

\tikzset{
	arrow/.style={-latex, shorten >=0pt, shorten <=0pt},
	arrowboth/.style={<->/.tip = Latex, shorten >=0pt, shorten <=0pt}
}

\usepackage{xcolor}

\definecolor{bgcolor0}{RGB}{255, 231, 231} %red!10
\definecolor{strokecolor0}{RGB}{255, 181, 181} %red!30

\definecolor{bgcolor1}{RGB}{231, 231, 255} %blue!10
\definecolor{bgcolor2}{RGB}{255, 243, 231} %orange!10

\definecolor{unired}{RGB}{165,0,0}
\definecolor{bgcolor3}{RGB}{238, 205, 205} %unired!20

\tikzset{
	cmdarrow/.style={-latex, shorten >=0pt, shorten <=0pt, color = unired},
	eventarrow/.style={-latex, shorten >=0pt, shorten <=0pt, color = blue!60},
	queryarrow/.style={-latex, shorten >=0pt, shorten <=0pt, color = blue!60},
	cmdcircle/.style={circle, draw, minimum size=5, inner sep=0pt, fill=unired},
	querycircle/.style={circle, draw, minimum size=5, inner sep=0pt, fill=blue!60},
	eventcircle/.style={circle, draw, minimum size=5, inner sep=0pt, fill=blue!60}
}

\definecolor{cmdlabel}{RGB}{165, 0, 0} 
\definecolor{eventlabel}{RGB}{102, 102, 255}
\definecolor{querylabel}{RGB}{102, 102, 255} 

\usepackage{tikz}

\begin{document}

\usetikzlibrary{arrows.meta}
\usetikzlibrary{positioning}
\usetikzlibrary{shapes.geometric}
\usetikzlibrary{calc} 
\tikzstyle{naveqs} = [text width=6em, fill=white, minimum height=5em, rounded corners, text centered, draw=black]
\tikzset{
	arrow/.style={-latex, shorten >=0pt, shorten <=0pt},
	arrowboth/.style={<->/.tip = Latex, shorten >=0pt, shorten <=0pt}
}

%-----#1 height, #2 width, #3 aspect, #4 name of the node, #5 coordinate, #6 label
\def\elBox[#1,#2,#3,#4,#5]#6{
	\node[draw, cylinder, alias=cyl, shape border rotate=90, aspect=#3, 
	minimum height=#1, minimum width=#2, outer sep=-0.5\pgflinewidth, 
	fill = white,
	%color=orange!40!black, left color=orange!70, right color=orange!80, middle color=white
	] (#4) at #5 {};

	%\node at #5 [text width=1.5cm, text centered] {#6};
	\node at #5 [text width=3.8cm, text centered] {#6};

	\fill [white] let \p1 = ($(cyl.before top)!0.5!(cyl.after top)$), 
		\p2 = (cyl.top), \p3 = (cyl.before top),
		\n1={veclen(\x3-\x1,\y3-\y1)},
		\n2={veclen(\x2-\x1,\y2-\y1)} in (\p1) ellipse (\n1 and \n2); 
}

%-----#1 height, #2 width, #3 aspect, #4 name of the node, #5 coordinate, #6 label
\def\aboxl[#1,#2,#3,#4,#5]#6{
	\node[draw, cylinder, alias=cyl, shape border rotate=90, aspect=#3, 
	minimum height=#1, minimum width=#2, outer sep=-0.5\pgflinewidth, 
	fill = white
	] (#4) at #5 {};

	\node at #5 [text width=1.5cm, text centered] {#6};

	\fill [white] let \p1 = ($(cyl.before top)!0.5!(cyl.after top)$), 
		\p2 = (cyl.top), \p3 = (cyl.before top),
		\n1={veclen(\x3-\x1,\y3-\y1)}, 
		\n2={veclen(\x2-\x1,\y2-\y1)} in (\p1) ellipse (\n1 and \n2); 
}

\begin{tikzpicture}[decoration=brace]

	\node (applogic) [naveqs, xshift = 52.5mm, minimum width = 20.5cm, text width = 5cm, fill = blue!10] at (-0.05, 1.3cm) {Application};

	\node (cqrs) [naveqs, label={[anchor=west]}, fill = red!10, minimum width = 20.5cm, minimum height = 8cm] at (5.20,-4.0cm) {};

	\node (api) [naveqs] {CQRS Application Interface};

	%\node (where) [align = center, yshift = 15mm, right = 10mm of applogic] {Where are the retroactive\\features available?};

	%\node [tri, shape border rotate=180] (A) {A};

	\node (commandmodel) [naveqs, below right = 10mm and 10mm of api] {Command Model};
	\draw [arrow, bend angle=25, bend left] 
		(api) to 
		node[midway,right=0.4em] {Commands}  
		(commandmodel);

	%\draw [arrow, bend angle=25, bend left] 
		%(commandmodel) 
		%node[midway, yshift=-25mm, xshift=15mm, left=0.0em, align = center] {Event\\Reference}  
		%to (api);

	%\elBox[5.2em,11.5em,2.0,eventlog, (11.7,-6.6)] { Command/Event Log + Retroactive Store };
	\elBox[5.2em,11.5em,2.0,eventlog, (11.7,-6.8)] { Timeline + Retroactive Store };
	\draw [arrow] 
		([yshift=0mm]commandmodel) to 
		node[midway, right=0.1em, text centered, 
		yshift=0mm, xshift=0mm, 
		text width = 2.5cm] {Commands\\\qquad\quad+ Events}  
		(eventlog);

	\node (querymodel) [naveqs,  below left = 10mm and 10mm of api, align = center] {Query\\Model};
	\draw [arrow, bend angle=25, bend right] 
		(api) to 
		node[midway,left=0.5em] {Queries}  
		(querymodel);

	\draw [arrow, bend angle=25, bend right] (querymodel) to (api);

	\aboxl[5.2em,5em,2.0,currentstate, (-3.4,-6.8)] { Current State };

	%\draw [<->,arrowboth] 
		%([yshift=-3mm]querymodel) to 
		%node[midway,right=0.1em, text centered, align = center] {Read-only\\Access}  
		%(currentstate);

	\draw [arrow] 
		(currentstate) to
		node[midway,right=0.0em, text centered, text width = 1.5cm] {Read-only\\Access}  
		([yshift=-8.7mm]querymodel);

	%\draw [arrow,  bend left, bend angle = 20, dashed] 
	\draw [arrow,   dashed] 
		([yshift=0mm]eventlog) to 
		node[midway,below=0.4em,text width=5.9cm, text centered] {State Updates (Events)}  
		([xshift=9mm]currentstate);

	%\path (api) -- (commandmodel) node[draw=none, midway, left]{ Commands };
	%\draw [bend angle=25, bend left] (api) to (commandmodel) {Commands};
	%\draw  (api)  to [bend left=45] node[sloped,midway,above] {midway label} (commandmodel) ;

	% Retroactive Stuff
	%\node (rapi) [naveqs, yshift = 40mm, right = 44mm of cqrs] {Retroactive Application Interface};

	\node (rcommandmodel) [naveqs, yshift = 0mm, right = 60mm of commandmodel] {Retroactive Operations Model};
	\draw [arrow, bend angle=8, bend left] 
		(commandmodel) to 
		node[midway, above=0.2em, align = center, text width = 4cm] {Retroactive\\Operations}  
		(rcommandmodel);
	\draw [arrow, bend angle=8, bend left] 
		(rcommandmodel) to 
		%node[midway, below=0.2em, align = center, text width = 4cm] {Event Reference}  
		(commandmodel);

	\node (rquerymodel) [naveqs,  below = 12mm of commandmodel, align = center] {Retroactive Query Model};
	\draw [arrow, bend angle=10, bend left] 
		(commandmodel) to 
		node[midway, left=1.6em, align = center] {Retroactive\\Queries}  
		(rquerymodel);

	\draw [arrow, bend angle=10, bend left] (rquerymodel) to (commandmodel);

	%\draw [arrow, bend angle=10, bend right] 
		%(rcommandmodel) to 
		%node[midway,below=0.7em,text width=2.2cm, text centered] {Events}  
		%(eventlog);

	%\draw [<->,arrowboth] 
	\draw [arrow] 
		(eventlog) to 
		node[near end,above=0.7em, text centered, align = center] {Read-only\\Access}  
		(rquerymodel);

	\draw [arrow] 
		([yshift=-3mm]rcommandmodel) to 
		node[midway,right=0.3em, align = center] {Write\\Access}  
		(eventlog);

	\draw[decorate, yshift=-4ex] (-1.5,-8) -- node[below=0.4ex] {eventually consistent} (-5,-8);

	\draw[decorate, yshift=-4ex] (14.5,-8) -- node[below=0.4ex] {always consistent} (2.0,-8);

\end{tikzpicture}

\end{document}
