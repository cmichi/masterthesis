\documentclass[tikz]{standalone}

\title{}
\author{Michael Müller}
\date{\today}

\usepackage{fontspec}
\usepackage{xunicode}
\usepackage{xltxtra}

\usepackage{polyglossia}
\setdefaultlanguage{english}

% big font for sections
\usepackage{sectsty}
\sectionfont{\LARGE}

\usepackage{lettrine}
\usepackage{graphicx}
\usepackage{wrapfig}
\usepackage{booktabs}
\usepackage{tabularx}
\usepackage[hidelinks]{hyperref}
\usepackage{listings}
\usepackage{pdfpages}
\usepackage{todonotes}

\usepackage[hang]{caption}
\usepackage{subcaption}

% enables us to use `\begin{figure}[H]`
\usepackage{float}

% \begin{comment} ... \end{comment{}
\usepackage{verbatim}


\makeatletter
\renewcommand{\paragraph}{
	\@startsection{paragraph}{4}
	{\z@}{1.25ex \@plus 1ex \@minus .2ex}{-1em}
	{\normalfont\normalsize\bfseries}
}
\makeatother

\usepackage{color}

\lstdefinestyle{stylea}{
	%basicstyle=\scriptsize\ttfamily,
	basicstyle=\scriptsize,
	tabsize=2,
	numbersep=5pt,
	firstline=1,
	showspaces=false,
	showstringspaces=false,
	xleftmargin=0pt
}

\lstdefinestyle{styleb}{
	numbers=left,
	frame=single,
	basicstyle=\scriptsize\ttfamily,
	tabsize=2,
	xleftmargin=0pt,
	showspaces=false,
	showstringspaces=false
}

\lstdefinestyle{stylec}{
	numbers=none,
	frame=none,
	basicstyle=\small\ttfamily,
	tabsize=2,
	%xleftmargin=0pt,
	showspaces=false,
	showstringspaces=false,
	,xleftmargin=.2\textwidth, xrightmargin=.2\textwidth
}

\lstdefinestyle{styled}{
	numbers=none,
	frame=none,
	basicstyle=\small\ttfamily,
	tabsize=2,
	showspaces=false,
	showstringspaces=false,
	xleftmargin=.1\textwidth
}

\usepackage{tikz}
\usetikzlibrary{fit,positioning}
\usetikzlibrary{decorations.text}

\usepackage{amssymb}

\usepackage[toc,page]{appendix}
\renewcommand{\appendixpagename}{\sffamily Appendices}

\usepackage[
        sorting=none,
        minbibnames=8,
        maxbibnames=9,
        block=space,
        backend=biber
]{biblatex}
\bibliography{bibliography}

\usepackage{paralist}

\defaultfontfeatures{
	Scale=MatchLowercase, % match the lowercase 'x' in different fonts
	Mapping=tex-text,
	%Ligatures={TeX, Common, Rare}, 
	Numbers={OldStyle, Proportional}
}
\setmainfont{Linux Libertine}
\setmonofont[Scale=0.82]{Source Code Pro}

\defaultfontfeatures[\sffamily]{
	Scale = MatchLowercase, 
	WordSpace = 0.6
}
\setsansfont{Latin Modern Sans}

\usepackage{mathtools}

\usepackage{fancyhdr}
% the roman numbers in the frontmatter somehow refuse to
% get centered on the bottom, thus i created a custom page 
% numbering  style for the abstract page
\fancypagestyle{page_number_bottom_center}{
	% clear all headers and footers
	\fancyhf{} 
	\fancyfoot[C]{\bfseries\thepage}

	% remove head rule
	\renewcommand{\headrulewidth}{0pt} 

	% remove foot rule
	\renewcommand{\footrulewidth}{0pt} 
}

% for character protrusion
\usepackage[protrusion=true]{microtype}

% to separate the final chapter in the table of contents from part 2
\usepackage{bookmark}

\newcommand{\cmd}[1]{\texttt{#1}}
\newcommand{\evt}[1]{\texttt{#1}}

\usepackage{scrtime}

\hyphenation{re-tro-action re-tro-active where-in}

% prevents page numbers and headings from 
% appearing on empty pages
\usepackage{emptypage}

\usepackage{tikz}

\begin{document}

\usetikzlibrary{arrows.meta}
\usetikzlibrary{positioning}
\usetikzlibrary{shapes.geometric}
\usetikzlibrary{calc} 
\tikzstyle{naveqs} = [text width=6em, fill=white, minimum height=5em, rounded corners, text centered, draw=black]
\tikzset{
	arrow/.style={-latex, shorten >=0pt, shorten <=0pt},
	arrowboth/.style={<->/.tip = Latex, shorten >=0pt, shorten <=0pt}
}

%-----#1 height, #2 width, #3 aspect, #4 name of the node, #5 coordinate, #6 label
\def\elBox[#1,#2,#3,#4,#5]#6{
	\node[draw, cylinder, alias=cyl, shape border rotate=90, aspect=#3, 
	minimum height=#1, minimum width=#2, outer sep=-0.5\pgflinewidth, 
	fill = white,
	%color=orange!40!black, left color=orange!70, right color=orange!80, middle color=white
	] (#4) at #5 {};

	%\node at #5 [text width=1.5cm, text centered] {#6};
	\node at #5 [text width=3.8cm, text centered] {#6};

	\fill [white] let \p1 = ($(cyl.before top)!0.5!(cyl.after top)$), 
		\p2 = (cyl.top), \p3 = (cyl.before top),
		\n1={veclen(\x3-\x1,\y3-\y1)},
		\n2={veclen(\x2-\x1,\y2-\y1)} in (\p1) ellipse (\n1 and \n2); 
}

%-----#1 height, #2 width, #3 aspect, #4 name of the node, #5 coordinate, #6 label
\def\aboxl[#1,#2,#3,#4,#5]#6{
	\node[draw, cylinder, alias=cyl, shape border rotate=90, aspect=#3, 
	minimum height=#1, minimum width=#2, outer sep=-0.5\pgflinewidth, 
	fill = white
	] (#4) at #5 {};

	\node at #5 [text width=1.5cm, text centered] {#6};

	\fill [white] let \p1 = ($(cyl.before top)!0.5!(cyl.after top)$), 
		\p2 = (cyl.top), \p3 = (cyl.before top),
		\n1={veclen(\x3-\x1,\y3-\y1)}, 
		\n2={veclen(\x2-\x1,\y2-\y1)} in (\p1) ellipse (\n1 and \n2); 
}

\begin{tikzpicture}[decoration=brace]

	\node (applogic) [naveqs, xshift = 52.5mm, minimum width = 20.5cm, text width = 5cm, fill = blue!10] at (-0.05, 1.3cm) {Application};

	\node (cqrs) [naveqs, label={[anchor=west]}, fill = red!10, minimum width = 20.5cm, minimum height = 8cm] at (5.20,-4.0cm) {};

	\node (api) [naveqs] {CQRS Application Interface};

	%\node (where) [align = center, yshift = 15mm, right = 10mm of applogic] {Where are the retroactive\\features available?};

	%\node [tri, shape border rotate=180] (A) {A};

	\node (commandmodel) [naveqs, below right = 10mm and 10mm of api] {Command Model};
	\draw [arrow, bend angle=25, bend left] 
		(api) to 
		node[midway,right=0.4em] {Commands}  
		(commandmodel);

	%\draw [arrow, bend angle=25, bend left] 
		%(commandmodel) 
		%node[midway, yshift=-25mm, xshift=15mm, left=0.0em, align = center] {Event\\Reference}  
		%to (api);

	%\elBox[5.2em,11.5em,2.0,eventlog, (11.7,-6.6)] { Command/Event Log + Retroactive Store };
	\elBox[5.2em,11.5em,2.0,eventlog, (11.7,-6.8)] { Timeline + Retroactive Store };
	\draw [arrow] 
		([yshift=0mm]commandmodel) to 
		node[midway, right=0.1em, text centered, 
		yshift=0mm, xshift=0mm, 
		text width = 2.5cm] {Commands\\\qquad\quad+ Events}  
		(eventlog);

	\node (querymodel) [naveqs,  below left = 10mm and 10mm of api, align = center] {Query\\Model};
	\draw [arrow, bend angle=25, bend right] 
		(api) to 
		node[midway,left=0.5em] {Queries}  
		(querymodel);

	\draw [arrow, bend angle=25, bend right] (querymodel) to (api);

	\aboxl[5.2em,5em,2.0,currentstate, (-3.4,-6.8)] { Current State };

	%\draw [<->,arrowboth] 
		%([yshift=-3mm]querymodel) to 
		%node[midway,right=0.1em, text centered, align = center] {Read-only\\Access}  
		%(currentstate);

	\draw [arrow] 
		(currentstate) to
		node[midway,right=0.0em, text centered, text width = 1.5cm] {Read-only\\Access}  
		([yshift=-8.7mm]querymodel);

	%\draw [arrow,  bend left, bend angle = 20, dashed] 
	\draw [arrow,   dashed] 
		([yshift=0mm]eventlog) to 
		node[midway,below=0.4em,text width=5.9cm, text centered] {State Updates (Events)}  
		([xshift=9mm]currentstate);

	%\path (api) -- (commandmodel) node[draw=none, midway, left]{ Commands };
	%\draw [bend angle=25, bend left] (api) to (commandmodel) {Commands};
	%\draw  (api)  to [bend left=45] node[sloped,midway,above] {midway label} (commandmodel) ;

	% Retroactive Stuff
	%\node (rapi) [naveqs, yshift = 40mm, right = 44mm of cqrs] {Retroactive Application Interface};

	\node (rcommandmodel) [naveqs, yshift = 0mm, right = 60mm of commandmodel] {Retroactive Operations Model};
	\draw [arrow, bend angle=8, bend left] 
		(commandmodel) to 
		node[midway, above=0.2em, align = center, text width = 4cm] {Retroactive\\Operations}  
		(rcommandmodel);
	\draw [arrow, bend angle=8, bend left] 
		(rcommandmodel) to 
		%node[midway, below=0.2em, align = center, text width = 4cm] {Event Reference}  
		(commandmodel);

	\node (rquerymodel) [naveqs,  below = 12mm of commandmodel, align = center] {Retroactive Query Model};
	\draw [arrow, bend angle=10, bend left] 
		(commandmodel) to 
		node[midway, left=1.6em, align = center] {Retroactive\\Queries}  
		(rquerymodel);

	\draw [arrow, bend angle=10, bend left] (rquerymodel) to (commandmodel);

	%\draw [arrow, bend angle=10, bend right] 
		%(rcommandmodel) to 
		%node[midway,below=0.7em,text width=2.2cm, text centered] {Events}  
		%(eventlog);

	%\draw [<->,arrowboth] 
	\draw [arrow] 
		(eventlog) to 
		node[near end,above=0.7em, text centered, align = center] {Read-only\\Access}  
		(rquerymodel);

	\draw [arrow] 
		([yshift=-3mm]rcommandmodel) to 
		node[midway,right=0.3em, align = center] {Write\\Access}  
		(eventlog);

	\draw[decorate, yshift=-4ex] (-1.5,-8) -- node[below=0.4ex] {eventually consistent} (-5,-8);

	\draw[decorate, yshift=-4ex] (14.5,-8) -- node[below=0.4ex] {always consistent} (2.0,-8);

\end{tikzpicture}

\end{document}
