\documentclass[tikz]{standalone}

\title{}
\author{Michael Müller}
\date{\today}

\usepackage{fontspec}
\usepackage{xunicode}
\usepackage{xltxtra}

\usepackage{polyglossia}
\setdefaultlanguage{english}

% big font for sections
\usepackage{sectsty}
\sectionfont{\LARGE}

\usepackage{lettrine}
\usepackage{graphicx}
\usepackage{wrapfig}
\usepackage{booktabs}
\usepackage{tabularx}
\usepackage[hidelinks]{hyperref}
\usepackage{listings}
\usepackage{pdfpages}
\usepackage{todonotes}

\usepackage[hang]{caption}
\usepackage{subcaption}

% enables us to use `\begin{figure}[H]`
\usepackage{float}

% \begin{comment} ... \end{comment{}
\usepackage{verbatim}


\makeatletter
\renewcommand{\paragraph}{
	\@startsection{paragraph}{4}
	{\z@}{1.25ex \@plus 1ex \@minus .2ex}{-1em}
	{\normalfont\normalsize\bfseries}
}
\makeatother

\usepackage{color}

\lstdefinestyle{stylea}{
	%basicstyle=\scriptsize\ttfamily,
	basicstyle=\scriptsize,
	tabsize=2,
	numbersep=5pt,
	firstline=1,
	showspaces=false,
	showstringspaces=false,
	xleftmargin=0pt
}

\lstdefinestyle{styleb}{
	numbers=left,
	frame=single,
	basicstyle=\scriptsize\ttfamily,
	tabsize=2,
	xleftmargin=0pt,
	showspaces=false,
	showstringspaces=false
}

\lstdefinestyle{stylec}{
	numbers=none,
	frame=none,
	basicstyle=\small\ttfamily,
	tabsize=2,
	%xleftmargin=0pt,
	showspaces=false,
	showstringspaces=false,
	,xleftmargin=.2\textwidth, xrightmargin=.2\textwidth
}

\lstdefinestyle{styled}{
	numbers=none,
	frame=none,
	basicstyle=\small\ttfamily,
	tabsize=2,
	showspaces=false,
	showstringspaces=false,
	xleftmargin=.1\textwidth
}

\usepackage{tikz}
\usetikzlibrary{fit,positioning}
\usetikzlibrary{decorations.text}

\usepackage{amssymb}

\usepackage[toc,page]{appendix}
\renewcommand{\appendixpagename}{\sffamily Appendices}

\usepackage[
        sorting=none,
        minbibnames=8,
        maxbibnames=9,
        block=space,
        backend=biber
]{biblatex}
\bibliography{bibliography}

\usepackage{paralist}

\defaultfontfeatures{
	Scale=MatchLowercase, % match the lowercase 'x' in different fonts
	Mapping=tex-text,
	%Ligatures={TeX, Common, Rare}, 
	Numbers={OldStyle, Proportional}
}
\setmainfont{Linux Libertine}
\setmonofont[Scale=0.82]{Source Code Pro}

\defaultfontfeatures[\sffamily]{
	Scale = MatchLowercase, 
	WordSpace = 0.6
}
\setsansfont{Latin Modern Sans}

\usepackage{mathtools}

\usepackage{fancyhdr}
% the roman numbers in the frontmatter somehow refuse to
% get centered on the bottom, thus i created a custom page 
% numbering  style for the abstract page
\fancypagestyle{page_number_bottom_center}{
	% clear all headers and footers
	\fancyhf{} 
	\fancyfoot[C]{\bfseries\thepage}

	% remove head rule
	\renewcommand{\headrulewidth}{0pt} 

	% remove foot rule
	\renewcommand{\footrulewidth}{0pt} 
}

% for character protrusion
\usepackage[protrusion=true]{microtype}

% to separate the final chapter in the table of contents from part 2
\usepackage{bookmark}

\newcommand{\cmd}[1]{\texttt{#1}}
\newcommand{\evt}[1]{\texttt{#1}}

\usepackage{scrtime}

\hyphenation{re-tro-action re-tro-active where-in}

% prevents page numbers and headings from 
% appearing on empty pages
\usepackage{emptypage}


\usepackage{tikz}

\begin{document}

\usetikzlibrary{arrows.meta}
\usetikzlibrary{positioning}
\usetikzlibrary{shapes.geometric}
\usetikzlibrary{calc} 
\tikzstyle{naveqs} = [text width=6em, fill=white, minimum height=5em, rounded corners, text centered, draw=black, minimum width=7.7cm]
\tikzset{
	arrow/.style={-latex, shorten >=0pt, shorten <=0pt},
	arrowboth/.style={<->/.tip = Latex, shorten >=0pt, shorten <=0pt}
}

%-----#1 height, #2 width, #3 aspect, #4 name of the node, #5 coordinate, #6 label
\def\elBox[#1,#2,#3,#4,#5]#6{
	\node[draw, cylinder, alias=cyl, shape border rotate=90, aspect=#3, 
	minimum height=#1, minimum width=#2, outer sep=-0.5\pgflinewidth, 
	fill = white!30,
	%color=orange!40!black, left color=orange!70, right color=orange!80, middle color=white
	] (#4) at #5 {};

	%\node at #5 [text width=1.5cm, text centered] {#6};
	\node at #5 [text width=1.9cm, text centered] {#6};

	\fill [white!30] let \p1 = ($(cyl.before top)!0.5!(cyl.after top)$), 
		\p2 = (cyl.top), \p3 = (cyl.before top),
		\n1={veclen(\x3-\x1,\y3-\y1)},
		\n2={veclen(\x2-\x1,\y2-\y1)} in (\p1) ellipse (\n1 and \n2); 
}
%-----ABoxes
%-----#1 height, #2 width, #3 aspect, #4 name of the node, #5
%-----coordinate, #6 label
\def\aboxl[#1,#2,#3,#4,#5]#6{%
	\node[draw, cylinder, alias=cyl, shape border rotate=90, aspect=#3, %
	minimum height=#1, minimum width=#2, outer sep=-0.5\pgflinewidth, %
	fill = white!30%
] (#4) at #5 {};%
	\node at #5 [text width=1.5cm, text centered] {#6};%
	\fill [white!30] let \p1 = ($(cyl.before top)!0.5!(cyl.after
	top)$), \p2 =
	(cyl.top), \p3 = (cyl.before top),
	\n1={veclen(\x3-\x1,\y3-\y1)},
	\n2={veclen(\x2-\x1,\y2-\y1)} in (\p1) ellipse (\n1 and
	\n2); }
		    

\begin{tikzpicture}
	\def\y{2}
	\def\x{2.7}
	\def\n{0.2}
	\def\shift{5mm}

	\node (commands) at (\x, 3) {Commands};
	\node (queries) at (-\x, 3) {Queries};

	\node (commandmodel) [naveqs, fill = white!30] at (0, 0){Model};
	%\draw [arrow] (commands) to (commandmodel);
	\draw [arrow] (\x + \n, 2.8) to (\x + \n, 0.89);

	%\draw [arrow] (queries) to (querymodel);

	%\draw [arrow] ([xshift=2cm]querymodel) to ([xshift=2cm]queries);
	% uebler hack
	\draw [arrow] (-\x - \n, 2.8) to (-\x - \n, 0.89);
	\draw [arrow] (-\x + \n, 0.89) to (-\x + \n, 2.8);

	%\draw [arrow] ([transform canvas={xshift=\shift}]querymodel) to ([transform canvas={xshift=\shift}]queries);

\end{tikzpicture}

\end{document}
