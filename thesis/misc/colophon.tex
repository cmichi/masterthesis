\cleardoublepage

\chapter*{Colophon}

This thesis was typeset using the \XeTeX{} engine. The microtype package is 
responsible for micro-typographic adjustments concerning the color of the text, 
line breakages, and the addition of character protrusion. TikZ has been used 
for the illustrations.
The font used for the text and headings is Philipp Poll's Linux Libertine. 
Some of the beautiful open type features which the typeface provides -- such 
as old-style figures and swashes -- have been activated. The font used for the 
monospaced code listings is Paul Hunt's \texttt{Source Code Pro}, which he 
developed for Adobe. Both fonts are provided under free licenses.\\

\noindent{}The quote used on page \pageref{page:quote} is taken from Grimm's 
Household Tales by Jacob and Wilhelm Grimm. Their collection of fairy tales 
was first published in 1812 and translated from German to English in 1884 by 
Margaret Hunt\footnote[1]{\href{https://en.wikisource.org/wiki/Page:Grimm\%27s_Fairy_Tales.djvu/355}{https://en.wikisource.org/wiki/Page:Grimm\%27s\_Fairy\_Tales.djvu/355}}. 
The text is part of the public domain. It is typeset in the Linux Libertine 
font as well, though with historical ligatures.\\

\noindent{}The title page typeface is Knuth's \textsf{Computer Modern Sans Serif}.
The illustration on the title page is titled »\emph{H\"ansel and Gretel}«;
it was created by Alexander Zick (1845-1907) and is in the public 
domain\footnote[2]{\href{https://commons.wikimedia.org/wiki/File:H\%C3\%A4nsel_und_Gretel.jpg}{https://commons.wikimedia.org/wiki/File:H\"ansel\_und\_Gretel.jpg 
}}.\\

\noindent{}The choice of the title illustration and the subsequent quote was
motivated by parallels in event-sourced systems and this certain fairytale. 
Pigeons ate their breadcrumb trail and thus H\"ansel and Grethel were unable 
to find their way back home. %which the siblings had created. 
The objective of an event-sourced system is to never get to a point where it 
is not possible to reconstruct how the system got there. 
This thesis shows that there is much more to an event-sourced system. Not only 
can we rebuild how we got somewhere, we can also explore which other houses we 
could have reached taking different trails through the forest. How the story 
could have gone differently, if H\"ansel and Gretel had had an event-sourced 
system with retroactive capabilities\dots{}
