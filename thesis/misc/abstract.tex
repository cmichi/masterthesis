\thispagestyle{page_number_bottom_center}
\section*{Abstract}

Event sourcing is a style of software architecture wherein state altering 
operations to an application are captured as immutable events. Each event 
is appended to an event log, with the current state of a system derived 
from this series of events. This thesis addresses the utilization of 
retroactive capabilities in event-sourced systems: computing alternate 
application states, post hoc bug fixes, or the support of algorithms which 
have access to their own history, for example.
The possibility of retroactively accessing and modifying this event log 
is a potential capability of an event-sourced system, but a detailed 
exploration how these operations can be facilitated and supported has not 
yet been conducted.

We examine how retroaction can be applied to event-sourced systems and 
discuss conceptual considerations. Furthermore, we demonstrate how different 
architectures can be used to provide retroaction and describe the prototypical 
implementation of an appropriate programming model.
These findings are applied in the Chronograph research project, in order to
utilize potential temporal aspects of this platform.
