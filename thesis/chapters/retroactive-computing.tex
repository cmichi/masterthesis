\section{Retroactive Computing}
In order to examine how retroactive capabilities can be utilized in event-sourced 
systems, we explicitly examine event sourcing \emph{in a CQRS context}. As described 
in Chapter~\ref{chp:background}, event sourcing and CQRS form a symbiotic relationship 
which is commonly taken advantage of. By assuming such a style of architecture, we 
can focus on how the event sourcing foundation can be utilized. This is not to say 
that we do not examine other architectures. 
Indeed, in Chapter \ref{chp:chrono} of this thesis we examine if and how the concepts 
described in this (and the succeeding) chapter can be applied to a different 
architecture.  The Chronograph platform is the foundation for this second examination; 
it applies event sourcing in a context which has only few commonalities with the 
traditional CQRS style of architecture.
