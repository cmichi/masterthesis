\chapter{Introduction}
\label{chp:introduction}

\section{Motivation}

\emph{Event sourcing (ES)} describes a style of architecture wherein each 
state altering operation in an application is captured as an immutable event 
\cite{Fowler2005}.
These events are appended to an \emph{event log}. Once appended they can
never be deleted, the event log possesses a write-once, append-only restriction.
The current state of an application is derived from this series of events.
When replayed, this series of events always leads to the exact same state 
of the application. 

Systems following such an event-sourced architecture have been gaining
popularity in recent years. There are a number of reasons for this; among
them is that event-driven architectures enable loosely coupled, composable
systems and possess an inherent asynchronous style \cite{Hohpe2006}.
As such they fit well for e.g. distributed systems with a need to scale.
%
A distributed architecture which applies event sourcing can utilize the 
immutable nature of events and the append-only semantics of the event log 
beneficial for scalability.
Event sourcing also fits well for systems with the requirement of an audit 
log, since events are immutable and the event log possesses an append-only 
restriction. It can always be clearly traced how a system reached its 
current state. 

This thesis focuses on the \emph{retroactive} aspect of event-sourced
systems -- exploring the past of existing or simulated event logs and the 
possibilities which can be gained hereby. 
Manipulations of the past contradict the append-only behavior of the event 
log, but can enable very interesting applications. Among other applications, 
the recorded history can be replayed in order to evaluate how other events 
or an alternate application behavior would have influenced the current state 
of the system \cite{Erb2014}.
These retroactive possibilities are potential capabilities of event-sourced 
systems, but a practical exploration of how these operations can be enabled 
and supported has not yet been conducted.

\section{Overview}
Restoring prior states of a system is already a well-established feature of 
event-sourced architectures. This thesis examines another very interesting 
aspect of this architectural style, which we think has not yet been considered 
sufficiently: 
%
Event sourcing is a perfect match for a computing model in which programmers 
can modify and interact with the application's history retroactively in the 
same environment in which the program runs. 

We aim to provide a thorough examination of the retroactive capabilities of 
event-sourced systems. This work identifies issues and constraints that need 
to be taken into consideration when utilizing these capabilities. Furthermore, 
we identify benefits which systems can gain from retroactive features. The 
practicality of our ideas is analyzed in a programming model and a prototypical 
implementation. In order to gain further insights into the applicability of our 
ideas, we apply them in an existing research project.

\section{Outline}
In Chapter \ref{chp:background}, we describe the terminology and background of 
event sourcing and describe work related to this thesis' objectives. 
%
Chapter \ref{chp:concept} focuses on a conceptual view on the topic, identifies 
challenges, and details our ideas. For this, we describe an event-sourced 
architecture following Command Query Responsibility Segregation (CQRS), a style 
of system architecture well-suited to event-sourced systems. Next, we examine 
how retroactive computing features can be integrated into such an architecture.
A number of issues and limitations, which need to be taken into account for 
this, are identified %These findings are used to derive two contrasting 
conceptual models. and we suggest two potential event-sourced architectures for 
retroactive computing.
%
In Chapter \ref{chp:programming}, we outline a programming model which utilizes 
retroactive capabilities of an ES+CQRS architecture and takes retroactive issues 
into account.
This programming model is implemented as a prototype with an example scenario.
Chapter \ref{chp:sum} summarizes our architectural considerations.
%
In Chapter \ref{chp:chrono}, we examine the applicability of our ideas in an 
event-sourced architecture that does not follow CQRS. The Chronograph research 
project is the foundation for this examination.
%
Chapter \ref{chp:results} provides our concluding remarks and closes the thesis.
