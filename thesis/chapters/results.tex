\chapter{Results and Conclusion}
\label{chp:results}

This thesis provides an extensive examination of the concept of retroactive 
computing in event-sourced systems. We identified key challenges, proposed 
solutions, and demonstrated their applicability. This chapter provides our 
concluding remarks.

%================
\section{Summary}
Event sourcing is a perfect match for retroactive computing, since it inherently 
captures state changes as modifications. Thus, prior states can be restored or 
retroactively modified.
%
The Chapters \ref{chp:background} and \ref{chp:related-work} described the field, 
related work, and delimitations of existing work to our objectives. In Chapter 
\ref{chp:concept} we identified major issues of retroaction, such as temporal 
and causal inconsistencies, or side effects. We illustrated that restraining 
direct editing operations to branches of a timeline can prevent causality issues. 
Furthermore, it is possible to retain traceability and append-only behavior in
event-sourced systems if results from retroaction are integrated back into the 
system as events.

Considering commands in replay scenarios, we discussed several possibilities of 
imposing control over potential side effects, and concluded by introducing partial 
replays, which require the tracking of causalities among events. 
The command and event primitives of event sourcing with CQRS are an ideal match 
for imposing control over side effects. Through events, side effects can be 
recorded and reused. Through the sourcing of commands, side effects can be 
reinvoked. If side effects are outsourced into separate, individual commands, 
they can be partially reused or reinvoked.


Next, we discussed a number of ideas for keeping a consistent timeline after a 
retroactive modification. For this, we transferred ideas from time travel theory 
to retroaction in event-sourced systems. Corresponding to parallel universe 
theories, we proposed the prohibition of editing the system's own timeline, while 
allowing direct retroactive modifications only for branches.
Analogous to the self-consistency principle, we proposed validation conditions 
and the replay (or removal) of causally related events. 
%
Furthermore, this chapter contributed an overview on constraints and limitations 
of retroaction in event-sourced systems. We highlighted the central role of 
hidden causalities through real-world coupling or through side effects. 
Further limitations, which we identified, were the influence of causality 
violations, causally equivalent replays, and the semantics of commands which 
may annihilate retroactive modifications. The performance of retroactive 
computations can be a further limiting factor.
%
Our conceptual considerations were concluded with the suggestion of architectural 
modifications (e.g. resolving the command/query segregation for retroaction) and 
the demonstration how different architectures can be used for enabling retroactive 
capabilities of event-sourced applications.
%
We argued that it is heavily dependent on the actual application domain how much 
can be made of retroaction and how high the informative value of retroactive 
modifications can be.
If retroaction is taken into account from the start when building a system, the 
informative value of retroactive changes can be maximized and limitations reduced.

Chapter \ref{chp:programming} demonstrated the practical applicability of the 
concepts described in Chapter \ref{chp:concept}. We outlined a programming model 
for the unified architecture. Moreover, a prototypical implementation was 
described using the scenario of an online shop. 
We demonstrated how issues of retroactive systems can be addressed in a
programming model (e.g. by constraining possible modifications) and which new 
challenges emerge from the application of our ideas (e.g. retroaction-aware 
programming). Furthermore, the central role of the delta encoding of events was 
highlighted for its effect on retroaction.
%
Our programming model provides the ability to access and manipulate the application's 
state history in a single environment. This enables a variety of possibilities, 
such as history-aware algorithms and novel debugging schemes.
Developers can use the application's data structures, functions, and libraries 
in the retroactive code. They can conduct analyses of the application's history
or explore alternate branches.

Integrating retroaction in event-sourced systems introduces a set of capabilities
which are not feasible in traditional event-sourced architectures. 
Chapter \ref{chp:sum} illustrated that a comparison of our conceptual and
architectural considerations to existing systems is difficult, since the 
capabilities which we propose for event-sourced systems cannot be found in this 
combination in the event-sourced field, nor in related work.

In Chapter \ref{chp:chrono}, we applied the ideas described in Chapters \ref{chp:concept} 
and \ref{chp:programming} to the Chronograph research platform.
Many of the described concepts can be applied there as well: timeline pointers 
via tagging, the delta encoding of events, retroaction-aware programming, or 
tracking causalities, for example.
Moreover, we proposed platform-specific solutions, such as I/O vertices or portals.
%
An additional element that can be transferred to Chronograph consists of the 
limitations of retroaction. These limitations pose a significant restriction 
to the expressiveness and informative value of retroaction. They can be reduced 
when taken into account while creating a system.


%===============================
\section{Outlook and Future Work}

Retroactive computing allows for self-improving, self-modifying, self-learning, 
and self-referential algorithms in event-sourced systems. 
Applications can adapt their future behavior by analyzing their own past. An area 
where this might have interesting applications is the usage of retroactive 
computing in event-sourced systems for machine learning, artificial intelligence,
or data mining. The application's history could be used as input to optimize 
parameters of an algorithm.
In a further step, updated or modified versions of algorithms could be evaluated 
and compared against each other by replaying commands from the timeline in 
experimental branches.

Future work on the topic should further analyze which domains could benefit of 
retroactive computing. For this, more event-sourced architectures which enable 
retroactive capabilities need to be taken into consideration, apart from the two 
architectures that we examined.

\pagebreak

\section{Conclusion}

To the best of our knowledge, we provide the first detailed examination of the 
concept of retroactive computing in event-sourced systems. 
The utilization of retroaction in event-sourced systems enables a set of features 
that cannot be accomplished with traditional application architectures. 
Applications cannot only analyze their own history retrospectively, they can also 
actively modify their past in order to explore alternative states.
%
We illustrated that the usage of retroactive computing is heavily dependent on 
the application domain and its domain-specific constraints. Some domains cannot 
take advantage of its full potential due to strict constraints caused by side 
effects, real-world coupling, or hidden causalities. Other domains on the other 
hand benefit heavily of retroactive aspects, as it allows for an entirely new 
perspective on application state.
